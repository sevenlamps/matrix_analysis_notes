\documentclass{article}
\usepackage[utf8]{inputenc}
\usepackage[english]{babel}
\usepackage{amssymb}
\usepackage{amsthm}
\makeatletter
\def\th@plain{%
  \thm@notefont{}% same as heading font
  \itshape % body font
}
\def\th@definition{%
  \thm@notefont{}% same as heading font
  \normalfont % body font
}
\makeatother
\usepackage{amsmath}
\usepackage{enumitem}
\usepackage{tikz-cd}
\usepackage[a4paper, total={6in, 8in}]{geometry}
% \setcounter{section}{-1}

% \usepackage{multicol}
% \setlength{\columnsep}{1cm}

\newtheorem*{lemma}{Lemma}
\newtheorem*{example}{Example}
\newtheorem*{remarks}{Remarks}
\newtheorem*{definition}{Definition}
\newtheorem{theorem}{Theorem}[section]
\newtheorem{proposition}[theorem]{Proposition}
\newtheorem{corollary}[theorem]{Corollary}
\newtheorem*{least_integer_axiom}{Least Integer Axiom\footnote{This property is usually called the \textit{well-ordering principle}}}

\begin{document}
    \section{Things Past}
    \subsection{Some Number Theory}
    \begin{definition}
        The set of \textbf{natural numbers} \(\mathbb{N}\) is defined by \[\mathbb{N}=\{integers \  n: n \geq 0\}\]
    \end{definition}
    \begin{least_integer_axiom}
        There is a smallest number \(n\) in every non-empty subset \(C\) of \(\mathbb{N}\).
    \end{least_integer_axiom}
    \begin{definition}
        A natural number is \textbf{prime} if \(p \geq 2\) and there is no factorization
        \(p=ab\) where \(a < p\) and \(b < p\) are natural numbers.
    \end{definition}
    \begin{proposition}
        Every integer \(n \geq 2\) is either a prime or a product of primes.
    \end{proposition}
    \begin{proof}
        Let \(C\) be the subset of \(\mathbb{N}\) consisting of all those \(n \geq 2\)
        for which the proposition is false. If \(C\) is non-empty, then there exists a 
        smallest number \(k\) in \(C\). Since \(k\) is not a prime, then there are 
        natural numbers \(a\) and \(b\) such that \(k=ab\), where \(a < k\) and \(b < k\). 
        But \(a\) and \(b\) are not in \(C\) since \(k\) is the smallest in \(C\), 
        then \(a\) and \(b\) are primes or product of primes. Therefore, the smallest number \(k\) 
        in \(C\) is a product of primes, contradicting the proposition.
    \end{proof}
    \begin{theorem}[Mathematical Induction]
        Let \(S(n)\) be a family of statements, one for each integer \(n \geq m\), where 
        \(m\) is some fixed number. If 
        \begin{enumerate}[label=(\roman*)]
            \item \(S(m)\) is true, and
            \item if \(S(n)\) is true implies \(S(n+1)\) is ture, 
        \end{enumerate}
        then \(S(n)\) is true for all integers \(n \geq m\).
    \end{theorem}
    \begin{proof}
        Let \(C\) be the set of all integers \(n \geq m\) for which \(S(n)\) is false. 
        If \(C\) is not empty, there is a smallest integer \(k\) in \(C\) such that \(S(k)\) 
        is false. By (i) we have \(k > m\), then there exists an integer \(k-1 \notin C\) 
        such that \(S(k-1)\) is true. By (ii), we have \(S((k-1)+1)=S(k)\), where \((k-1)+1=k \notin C\)
        is also true. This contradicts the assumption that \(C\) is non-empty, thus \(C\)
        is empty. Therefore, the proposition is true. 
    \end{proof}
    \begin{theorem}[Second Form of Induction]
        Let \(S(n)\) be a family of statements, one for each integer \(n \geq m\), where 
        \(m\) is some fiexed integer. If 
        \begin{enumerate}
            \item \(S(m)\) is true, and
            \item if \(S(k)\) is true for all \(k\) with \(m \leq k < n\), then \(S(n)\) is itself true,
        \end{enumerate}
        then \(S(n)\) is true for all integers \(n \geq m\).
    \end{theorem}
    \begin{proof}
        Let \(C\) be the set of all integers \(n \geq m\) for which \(S(n)\) is false. 
        If \(C\) is not empty, there is a smallest integer \(k\) in \(C\) such that \(S(k)\) 
        is false. By (i) we have \(k > m\), then there exists an integer \(k-1 \notin C\) 
        such that \(S(k-1)\) is true. Then by (ii), since \(S(i)\) is true for all \(i\) 
        with \(m \leq i < k\), then \(S(k)\) is itself true, contradicting the assumption 
        that \(S(k)\) is false.
    \end{proof}
    \begin{theorem}[Division Theorem]
        Given integers \(a\) and \(b\) with \(a \neq 0\), there exist unique integers 
        \(q\) and \(r\) with \[b=qa+r \ \ and \ \ 0 \leq r < |a|\]
    \end{theorem}
    \begin{proof}
        Suppose there exist another pair of integers \(q'\) and \(r'\) with \(b=q'a+r'\) 
        where \(0 \leq r' < |a|\). Then \(qa+r=q'a+r' \implies |(q-q')a|=|r'-r|\). 
        Since \(0 \leq |r'-r| < |r'| < |a| \implies 0 \leq |(q-q')a| < |a|\), if \(a>0\), 
        then \(0 \leq |q-q'| < 1\), recall that \(q\) and \(q'\) are both integers, 
        then \(q=q'\); if \(a<0\), then \(-1 < |q-q'| \leq 0 \implies q=q'\). Both cases 
        implies \(r=r'\) as well. This contradicts the assumption, therefore, the integers 
        are unique.
    \end{proof}
    \begin{definition}
        If \(a\) and \(b\) are integers with \(a \neq 0\), then the integers \(q\) and 
        \(r\) occurring in the division algorithm are called \textbf{quotient} and 
        \textbf{remainder} after dividing \(b\) by \(a\).
    \end{definition}
    \begin{corollary}
        There are infinitely many primes.
    \end{corollary}
    \begin{proof}
        (\textbf{\textit{Euclid}})  Suppose there are k finite primes \(p_1, p_2, \cdots, p_k\).
        Then define \(M=\prod_{i=1}^k p_i + 1\), by Proposition 1.1, it is either a 
        prime or a product of primes. Since our assumption indicates \(M\) is not a 
        prime, then it must be a product of primes. But the fact that \(\dfrac{M}{\prod_{i=1}^k p_k}\) 
        gives remainder not 0 but 1 shows \(M\) cannot be divided by the existing 
        product of primes, by definition, \(M\) is a prime, which contradicting the 
        assumption. So there must be infinite number of primes.
    \end{proof}
    \begin{definition}
        If \(a\) and \(b\) are integers, then \(a\) is a \textbf{divisor} of \(b\) if 
        there is an integer \(d\) with \(b=ad\). We also say that \(a\) \textbf{divides} 
        \(b\) or that \(b\) is a \textbf{multiple} of \(a\), and we denote this by \(a \mid b\)
    \end{definition}
    \begin{definition}
        A \textbf{common divisor} of integers \(a\) and \(b\) is an integer \(c\) with 
        \(c \mid a\) and \(c \mid b\). The \textbf{greatest common divisor} or \textbf{gcd} 
        of \(a\) and \(b\), denoted by \((a, b)\), is defined by 
        \begin{equation*}
            (a, b) = \begin{cases}
                0 \ if \  a=0=b\\
                the\ largest\ common\ divisor\ of\ a\ and\ b\ otherwise
            \end{cases}
        \end{equation*}
    \end{definition}
    \begin{proposition}
        If \(p\) is a prime and \(b\) any given integer, then 
        \[
            (p, b)=\begin{cases}
                p\ if\ p \mid b\\
                1\ otherwise
            \end{cases}  
        \]
    \end{proposition}
    \begin{proof}
        Since \(p\) is a prime, i.e., \(p=p \cdot 1 \) then \((p, p) = p\). If \(p \mid b\), 
        then we have \(p \mid p\) and \(p \mid b\) thus \((p, b)=p\); otherwise, if \(p \nmid b\), 
        then we have \(1 \mid p\) and \(1 \mid b\) thus \((p, b)=1\).
    \end{proof}
    \begin{theorem}
        If \(a\) and \(b\) are integers, then \((a, b)=d\) is a linear combination of \(a\)
        and \(b\); that is, there are integers \(s\) and \(t\) with \(d=sa+tb\).
    \end{theorem}
    \begin{proof}
        Since \((a,b)=d\), by division algorithm, we have \(a=dq_1\) and \(b=dq_2\) where 
        \(q_1,\ q_2 \in \mathbb{Z}\). Let \(s=\dfrac{q_1}{2}\) and \(t=\dfrac{q_2}{2}\), 
        then \(d=\dfrac{1}{2} \cdot (\dfrac{a}{q_1} + \dfrac{b}{q_2})\)
    \end{proof}
\end{document}